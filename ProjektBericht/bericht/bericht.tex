\documentclass{chi-ext}
% Please be sure that you have the dependencies (i.e., additional LaTeX packages) to compile this example.
% See http://personales.upv.es/luileito/chiext/

\copyrightinfo{
  Copyright is held by the author/owner(s).\\
  This is a generic SIGCHI \LaTeX\ template sample.\\
  The corresponding ACM copyright statement must be included.
}

\title{HCI-Projektbericht Draw-to-Clipboard}

\numberofauthors{8}
% Notice how author names are alternately typesetted to appear ordered in 2-column format;
% i.e., the first 4 autors on the first column and the other 4 auhors on the second column.
% Actually, it's up to you to strictly adhere to this author notation.
\author{
  \vspace{-1.5em} % lisatolles: The abstract heading should start at the time height on the page as the authors names
  \alignauthor{
  	\textbf{Constantin Gerstberger}\\
  	\affaddr{Theresienstr. 11}\\
  	\affaddr{82131 Gauting, Germany}\\
  	\email{constantin.gerstberger@gmail.com}
  }\alignauthor{
  	\textbf{Sebastian W\"ohrl}\\
  	\affaddr{123 Author Ave.}\\
  	\affaddr{Authortown, PA 54321 USA}\\
  	\email{author5@anotherco.com}
  }
  \vfil
  \alignauthor{
  	\textbf{Manfred Schmidbartl}\\
  	\affaddr{123 Author Ave.}\\
  	\affaddr{Authortown, PA 54321 USA}\\
  	\email{author2@anotherco.com}
  }
  \vfil
  \alignauthor{
  	\textbf{Benjamin Schwartz}\\
  	\affaddr{123 Author Ave.}\\
  	\affaddr{Authortown, PA 54321 USA}\\
  	\email{author3@anotherco.com}
  }
  \vfil
  \alignauthor{
  	\textbf{Marcus Vetter}\\
  	\affaddr{Hofheimerstr. 6}\\
  	\affaddr{81245 Muenchen, Germany}\\
  	\email{marcus.vetter@tum.de}
  }
}

% Paper metadata (use plain text, for PDF inclusion and later re-using, if desired)
\def\plaintitle{HCI-Projektbericht Draw-to-Clipboard}
\def\plainauthor{Luis A. Leiva}
\def\plainkeywords{Guides, instructions, author's kit, conference publications}
%\def\plaingeneralterms{Documentation, Standardization}

\hypersetup{
  % Your metadata go here
  pdftitle={\plaintitle},
  pdfauthor={\plainauthor},  
  pdfkeywords={\plainkeywords},
  %pdfsubject={\plaingeneralterms},
  % Quick access to color overriding:
  %citecolor=black,
  %linkcolor=black,
  %menucolor=black,
  %urlcolor=black,
}

\usepackage{graphicx}   % for EPS use the graphics package instead
\usepackage{balance}    % useful for balancing the last columns
\usepackage{bibspacing} % save vertical space in references


\begin{document}

\maketitle

\begin{abstract}
In this sample we describe the formatting requirements for various SIGCHI related submissions 
and offer recommendations on writing for the worldwide SIGCHI readership. 
%Do not change the page size or page settings.
Please review this document even if you have submitted to SIGCHI conferences before, 
some format details have changed relative to previous years.
\end{abstract}

\keywords{\plainkeywords}



% =============================================================================
\section{Problemstellung und Motivation}
% =============================================================================
This format is to be used for submissions that are published in the conference extended abstracts.  
We wish to give this volume a consistent, high-quality appearance. 
We therefore ask that authors follow some simple guidelines. 
In essence, you should format your paper exactly like this document. 
The easiest way to do this is simply to download a template from the conference website and replace the content with your own material.

% =============================================================================
\section{Low-fidelity Prototyp}
% =============================================================================
This format is to be used for submissions that are published in the conference extended abstracts.  
We wish to give this volume a consistent, high-quality appearance. 
We therefore ask that authors follow some simple guidelines. 
In essence, you should format your paper exactly like this document. 
The easiest way to do this is simply to download a template from the conference website and replace the content with your own material.

% =============================================================================
\section{Ergebnisse der Studie}
% =============================================================================
This format is to be used for submissions that are published in the conference extended abstracts.  
We wish to give this volume a consistent, high-quality appearance. 
We therefore ask that authors follow some simple guidelines. 
In essence, you should format your paper exactly like this document. 
The easiest way to do this is simply to download a template from the conference website and replace the content with your own material.

% =============================================================================
\section{Diskussion}
% =============================================================================
This format is to be used for submissions that are published in the conference extended abstracts.  
We wish to give this volume a consistent, high-quality appearance. 
We therefore ask that authors follow some simple guidelines. 
In essence, you should format your paper exactly like this document. 
The easiest way to do this is simply to download a template from the conference website and replace the content with your own material.

% =============================================================================
\section{Konklusion}
% =============================================================================
This format is to be used for submissions that are published in the conference extended abstracts.  
We wish to give this volume a consistent, high-quality appearance. 
We therefore ask that authors follow some simple guidelines. 
In essence, you should format your paper exactly like this document. 
The easiest way to do this is simply to download a template from the conference website and replace the content with your own material.



\balance
\bibliographystyle{acm-sigchi}
\bibliography{sample}

\end{document}